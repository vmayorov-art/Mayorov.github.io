Провести повне дослідження і побудувати графік функції
y(x)=(x^2+8)/(1-x)
	Область визначення функції. Так як функція являє собою дріб, потрібно знайти нулі знаменника.
1−x=0,  ⇒   x=1.

Виключаємо єдину точку x = 1 з області визначення функції і отримуємо:

D(y) = (−∞;1) ∪ (1;+∞).

	Досліджуємо поведінку функції в околі точки розриву. Знайдемо односторонні межі:

〖(lim⁡y)┬(x→1-0)=〗⁡lim┬(x→1-0)⁡〖(x^2+8)/(1-x)=+∞〗 

〖(lim⁡y)┬(x→1+0)=〗⁡lim┬(x→1+0)⁡〖(x^2+8)/(1-x)〗 =-∞


Так як межі рівні нескінченності, точка x = 1 є розривом другого роду, пряма x = 1 - вертикальна асимптота.

	Визначимо точки перетину графіка функції з осями координат.
Знайдемо точки перетину з віссю ординат Oy, для чого прирівнюємо x = 0:

y=(0^2+8)/(1-0)=8


Таким чином, точка перетину з віссю Oy має координати (0; 8). Знайдемо точки перетину з віссю абсцис Ox, для чого покладемо y = 0:


(x^2+8)/(1-x)=0→x^2+8=0


Рівняння не має коренів, тому точок перетину з віссю Ox немає. 


Зауважимо, що x^2+8>0 для будь-яких x. Тому при x ∈ (-∞; 1) функція y > 0 (набуває додатних значень, графік знаходиться вище осі абсцис), при x ∈ (1; + ∞) функція y < 0 (набуває від'ємних значень, графік знаходиться нижче осі абсцис).

	Функція не є ні парною, ні непарною, так як:
y(-x)=((〖-x〗^2 )+8)/(1-(-x))=(x^2+8)/(1+x)
При y(-x)≠y(x),y(-x)≠-y(x)
	Досліджуємо функцію на періодичність. Функція не є періодичною, тому що являє собою дрібно-раціональну функцію.
	Досліджуємо функцію на екстремуми і монотонність. Для цього знайдемо першу похідну функції:
y^'=(〖(x〗^2+8)')/((1-x))=(〖(x〗^2+8)'(1-x)-(x^2+8)(1-x)')/〖(1-x)〗^2 =(2x(1-x)-〖(x〗^2+8)(-1))/〖(1-x)〗^2 
=  (2x-〖2x〗^2+x^2+8)/〖(1-x)〗^2 =-(x^2-2x-8)/〖(1-x)〗^2 
Прирівняємо першу похідну до нуля і знайдемо стаціонарні точки (в яких y '= 0):
y^'=0→-(x^2-2x-8)/(1-x)^2 =0→x^2-2x-8=0→x=-2;x=4. 
Отримали три критичні точки: x = -2, x = 1, x = 4. Розіб'ємо всю область визначення функції на інтервали даними точками і визначимо знаки похідної в кожному проміжку:


                      -2                1                4

При x∈ (-∞; -2), (4; + ∞) похідна y '<0, тому функція спадає на даних проміжках.
При x∈ (-2; 1), (1; 4) похідна y '> 0, функція зростає на даних проміжках.
При цьому x = -2 - точка локального мінімуму (функція спадає, а потім зростає), x = 4 - точка локального максимуму (функція зростає, а потім зменшується).
Знайдемо значення функції в цих точках:
y(-2)=(〖-2〗^2+8)/((1-(-2) )=12/3=4y(4)=(4^2+8)/(1-4)=24/(-3)=-8

Таким чином, точка мінімуму (-2; 4), точка максимуму (4; -8).
	Досліджуємо функцію на перегини і опуклість. Знайдемо другу похідну функції:

y^''=((〖-x〗^2-2x-8)^'/(1-x)^2 )=-(〖(x〗^2-2x-8)'(〖1-x)〗^2-(x^2-2x-8)(〖(〖1-x)〗^2 )^'〗^'')/(1-x)^4 =

=-((2x-2)(〖1-x)〗^2-(x^2-2x-8)*2((1-x)*(-1))/(1-x)^4 =
=-((2x-2)(1-x)+2(x^2-2x-8))/(1-x)^3 =

=-2(〖-x〗^2+2x-1+x^2-2x-8)/(1-x)^3 =(-2*(-9))/(1-x)^3 =18/〖(1-x)〗^3 
Прирівняємо другу похідну до нуля:

y^''=0→18/〖(1-x)〗^3 =0
Отримане рівняння не має коренів, тому точок перегину немає. При цьому, коли 
x∈ (-∞; 1) виконується y ''> 0, тобто функція увігнута, коли x∈ (1; + ∞) виконується y '' <0, тобто функція опукла.

	Досліджуємо поведінку функції на нескінченності, тобто при x→±∞.
(lim⁡y=)┬(x→+∞)⁡lim┬(x→+∞)⁡〖(x^2+8)/(1-x)=lim┬(x→+∞)⁡〖x/(-1)=-∞〗 〗 

〖lim┬(x→-∞) y=〗⁡lim┬(x→-∞)⁡〖(x^2+8)/(1-x)=lim┬(x→-∞)⁡〖x/(-1)=+∞〗 〗 
Так як межі нескінченні, горизонтальних асимптот немає.
Спробуємо визначити похилі асимптоти виду y = kx + b. Обчислюємо значення k, b за відомими формулами:
k=lim┬(x→∞)⁡〖y/x〗=(x^2+8)/〖x-x〗^2 =1/(-1)=-1
b=lim┬(x→∞)⁡〖(y-kx)=lim┬(x→∞)⁡〖((〖(x〗^2+8)/(1-x)+x)=lim┬(x→∞)⁡〖((x^2+8+x-x^2)/(1-x))=〗 〗 〗

=lim┬(x→∞)⁡〖((8+x)/(1-x))=1/(-1)〗=-1
Отримали, що e функції є одна похила асимптота y = -x-1.
	Додаткові точки. Обчислимо значення функції в деяких інших точках, щоб точніше побудувати графік.
 y(-5)=5.5      y(2)=-12      y(7)=-9.5
	За отриманими даними побудуємо графік, доповнимо його асимптотами x = 1 (синій), y = -x-1 (зелений) і відзначимо характерні точки (фіолетовим перетин з віссю ординат, помаранчевим екстремуми, чорним додаткові точки):

